\documentclass[portugues]{./ppgccufmg}

\usepackage[brazil]{babel}
\usepackage{natbib}
\usepackage{xcolor}
\usepackage{lipsum}
\usepackage[
	colorlinks=true,
	linkcolor=blue,
	citecolor=red,
	urlcolor=magenta,
]{hyperref}

\begin{document}
	\ppgccufmg{
		autor={Eduardo, Iago, Julia, Julio},
		titulo={MOSFETs de Potência},
		cidade={Belo Horizonte},
		ano={2024},
		versao={Final},
		orientadora={},
		coorientadora={},
		resumo={resumo.tex},
		abstracten={abstract.tex},
		palavraschave={Matemática. Computação.},
		keywords={Math. Computing.},
		dedicatoria={dedicatoria.tex},
		agradecimentos={agradecimentos.tex},
		epigrafe={},
		epigrafeautor={},
		listadefiguras={sim},
		listadetabelas={sim},
	}
	
	\chapter{O MOSFET}
		Aqui, podemos dar uma breve introdução sobre MOSFETs, falando sobre aplicações comuns.
		
		\section{Funcionamento}
			Aqui, vamos explicar, a nível de materiais semicondutores, como funciona um MOSFET. Basicamente, revisar o que foi dado em aula.
			
			\subsection{Sub-seção}
			se precisar
				\subsubsection{Uma sub-sub-seção}
					idem

		\section{Parâmetros}
			Imagino que convenha usar uma tabela.

			\begin{table}[ht]
				\centering
				\begin{tabular}{|c|c|c|c|c|}
				\hline
								& I & V & P & T\textsubscript{j} \\ \hline
				Típico	&   &   &   &										 \\ \hline
				Máximo	&   &   &   &										 \\ \hline
				\end{tabular}
				\caption{Parâmetros para MOSFETs convencionais}
				\label{tab:mosfet_convencional_param}
			\end{table}

		\section{Implicações de Ultrapassar os Parâmetros Máximos}
			Explicar, a nível de semicondutores, o que acontece.

	\chapter{O MOSFET de Potência}
			Aqui, damos um geral sobre a criação do MOSFET de potência.

		\section{Características Eletrônicas}
			Explicar os aspectos construtivos do MOSFET de potência.

		\section{Funcionamento}
			Aqui, vamos explicar, a nível de materiais semicondutores, como funciona um MOSFET de potência.

		\section{Parâmetros}

			\begin{table}[ht]
				\centering
				\begin{tabular}{|c|c|c|c|c|}
				\hline
								& I & V & P & T\textsubscript{j} \\ \hline
				Típico	&   &   &   &										 \\ \hline
				Máximo	&   &   &   &										 \\ \hline
				\end{tabular}
				\caption{Parâmetros para MOSFETs de potência}
				\label{tab:mosfet_potencia_param}
			\end{table}

		\renewcommand\bibname{Referências}
		\bibliographystyle{plain}
		\bibliography{referencias}
		
\end{document}
